\documentclass{article}

% Language setting
% Replace `english' with e.g. `spanish' to change the document language
\usepackage[english]{babel}

% Set page size and margins
% Replace `letterpaper' with `a4paper' for UK/EU standard size
\usepackage[letterpaper,top=2cm,bottom=2cm,left=3cm,right=3cm,marginparwidth=1.75cm]{geometry}

% Useful packages
\usepackage{amsmath}
\usepackage{graphicx}
\usepackage[colorlinks=true, allcolors=blue]{hyperref}



\begin{document}
\date{} % Remove a data padrão
\begin{titlepage}
    \begin{center}
        \LARGE FATEC BAIXADA SANTISTA \\ Curso de Ciência de Dados\\
        
        \vspace{8cm}
        
        \LARGE Relatório: Análise de Componentes Principais (PCA) no Conjunto de Dados de Emprego formal-Base2020atual do SEADE\\
        
        \vspace{8cm}
        
        \large Autores: João Vitor Ferreira Lima\\
        João Victor Ruas Blum Lima Barboza\\
        Denis Holanda Medeiros de Araújo \\
Professor Orientador: Alexandre Garcia de Oliveira \\
        
        \vspace{2cm}
        
        \large Data:13/06/2023
    \end{center}
\end{titlepage}
\maketitle


\section{Introdução}

No seguinte relatório, será apresentada uma análise utilizando a técnica de Análise de Componentes Principais (PCA) aplicada a uma amostra de um conjunto de dados de Emprego Formal do Estado de São Paulo, provindo do site da Fundação SEADE. O objetivo é realizar uma redução de dimensionalidade e explorar as principais características do conjunto de dados.

\section{Descrição do Conjunto de Dados}

O conjunto de dados que está sendo utilizado neste estudo é o "Emprego Formal Base2020atual", obtido a partir do site da Fundação SEADE. Contendo informações concretas dos numeros de admissões, desligamentos e saldo movimentação, divididos assim por municipios, seção de atuação no mercado de trabalho e competencia (dado referente ao ano e mês de retirada da amostra), consistindo numa amostragem do inicio de 2020 até o final de 2023.

\section{Dimensão dos Dados}

O conjunto de dados possui 1.015.512 instâncias (amostras) que foram contabilizadas para a criação deste relatório, com o número de amostras sendo reduzidas à 150 para a análise, que incluem os numeros de admissões, desligamentos e saldo, sendo representados pelos períodos de 2020, 2021, 2022 e 2023. Portanto, o número de amostras é de 150, sendo divididas em aproximadamente 38 amostras por período.


\subsection{Visualização das primeiras linhas do conjunto de dados}
A figura abaixo são dos dados antes da normalização e tratamento dos valores ausentes.
\begin{figure}
\centering
\includegraphics[width=0.8\linewidth]{tabela.png}
\caption{\label{fig:frog}Visualização das primeiras linhas antes do tratamento pelo Google Colab}
\end{figure}

\newpage
\subsection{Matriz de covariância}

A matriz de covariância é uma medida estatística que descreve a relação entre as variáveis do conjunto de dados. Ela é calculada para identificar as correlações entre as características. 
Abaixo esta uma da Matriz Quadrada Reduzida da Matriz de Covariância:
\[
\begin{bmatrix}
1.00671141 & 1.00645381 & 0.9599312 \\
1.00645381 & 1.00671141 & 0.95282429 \\
0.9599312 & 0.95282429 & 1.00671141 \\
\end{bmatrix}
\]

\subsection{Autovalores e Autovetores}

Os autovalores e autovetores são calculados a partir da matriz de covariância e fornecem informações sobre as direções principais dos dados e suas importâncias relativas. Os autovetores são vetores que definem as direções principais (componentes principais) dos dados, enquanto os autovalores correspondentes indicam a variância explicada por cada componente principal.
Resultados dos dois maiores Autovalores e Autovetores:\\

Autovalores:
\[
\begin{bmatrix}
2.95304681e^\text{+00} & 2.68111178e^\text{-16} & 6.70874176e^\text{-02}
\end{bmatrix}
\]

Dois maiores autovalores:
\[
\begin{bmatrix}
2.9530468105922063\\
0.06708741759571583
\end{bmatrix}
\]

Autovetores:
\[
\begin{bmatrix}
-0.581356 &  -0.7300681 & -0.35920157\\
-0.57998252 & 0.68145642 & -0.44636018\\
-0.57065355 & 0.05116354 & 0.81959552
\end{bmatrix}
\]

Dois maiores autovetores:
\[
\begin{bmatrix}
-0.581356 & -0.57998252 & -0.57065355\\
-0.35920157 & -0.44636018 & 0.81959552
\end{bmatrix}
\]




\section{Plotagem}
A plotagem está logo após a análise dos resultados obtidos. 

\subsection{Resultados}

A plotagem resultante do PCA mostrou os pontos verdes (correspondentes ao ano de 2022) se mantém concentrados e próximos uns dos outros, tendo em conjunto a grande maior parte dos outros pontos(aos quais se referem aos anos de 2020, 2021 e 2023), porém pode-se notar que o grupo de pontos vermelhos e azuis(anos de 2021 e 2023 respectivamente), estão mais dispersos e têm poucas sobreposições com os demais pontos, junto de poucos exemplares dos pontos amarelos e verdes. Desta forma, é sugerido que as características calculadas têm uma forte aproximação dos dados de regressão, assim perrmitindo pensar que entre todo o período analisado, seus numeros gerais tendem a ser mais semelhantes entre si ao que se refere dos anos de 2020 e 2022, do que poucos resultados revelados dos anos de 2021 e 2023.

\subsection{Visualização do gráfico PCA do conjunto de dados}
A figura abaixo é o grafico com a analise feita.
\begin{figure}
\centering
\includegraphics[width=0.8\linewidth]{download.png}
\caption{\label{fig:frog}Imagem gerada pelos autores fornecida do Google Colab}
\end{figure}

As características utilizadas para a análise incluem proximidade dos valores em si, embora mudança contínua dos anos, visão geral de certos destaques da amostra total . Essas características foram calculadas com base nos índices e praticidade de entendê-los e foram selecionadas por sua relevância na visão de como está o mercado de trabalho que vivenciamos.

Essa análise pode ser útil para auxiliar no diagnóstico de câncer de mama, uma vez que fornece insights sobre as características mais discriminativas entre os diferentes tipos de tumores. Através da redução de dimensionalidade proporcionada pelo PCA, é possível obter uma representação visual dos dados, o que pode facilitar a interpretação e tomada de decisões.



\pagebreak


\bibliographystyle{alpha}
\bibliography{sample}
O código-fonte do projeto está disponível no GitHub do seguinte link:\\ \url{https://github.com/JVG4M35/Entropia/blob/main/PCA\_EMPREGO\_FINAL.ipynb}.\\
O link para o dataset usado neste relatório:\\ \url{https://repositorio.seade.gov.br/dataset/emprego-formal-2021-painel/resource/cb4131ce-e6db-40b4-9335-dfc7f61355e3}.
\end{document}